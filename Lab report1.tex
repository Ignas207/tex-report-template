\documentclass{article}
\usepackage{times}
\usepackage{amsmath} % Advanced math typesetting
\usepackage[utf8]{inputenc} % Unicode support (Umlauts etc.)
%\usepackage[ngerman]{babel} % Change hyphenation rules
\usepackage[colorlinks, urlcolor=blue, linkcolor=black]{hyperref} % Add a link to your document
\usepackage{graphicx} % Add pictures to your document
\usepackage{listings} % Source code formatting and highlighting
\graphicspath{{./resources/}}
\begin{document}
\author{Ignas Brazauskas}
\title{Lab 1 report}
\maketitle
\newpage

\tableofcontents{}
\newpage

\section{Introduction}
%Enter text here for the introduction part

\newpage

\section{Experiment procedure}
%Text here

\begin{table}[ht]
    \centering
    \begin{tabular}{c c c}
        \hline
        Rated voltage & $U_N$ & 480 $V$ \\
        \hline
        Rated current & $I_N$ & 270 $A$\\
        \hline
        Rated speed & $n_N$ & 3140 $r/min$\\
        \hline
        Rated torque & $T_n$ & 369 $Nm$\\
        \hline
        Armature resistance & $R_a$ & 0,045 $\Omega$\\
        \hline
        Armature inductance & $L_a$ & 4,5 $mH$\\
        \hline
        Flux constant & $k_f$ & 0,35 Vs\\
        \hline
        Total moment of inertia & $J$ & 0,35 $kgm^2$\\
        \hline
    \end{tabular}
    \caption{Rating and parameters of the DC motor}
    \label{motor paremeters}
\end{table}

\begin{figure}[h!]
    %\includegraphics[width=\linewidth]{dc motor with voltage limiter.png}
    \caption{DC motor simulation with 480V/0,1s voltage limiter}
    \label{dcmotor:with-v-limit}
\end{figure}

\newpage

\section{Experimental results}
%

\newpage

\section{Conclusions}

\newpage

\section{References}


\end{document}